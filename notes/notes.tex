\documentclass{article}

\usepackage{amsmath}
\usepackage{listings}
\lstset{language=lisp}

\title{Notes on {\tt libfpoly}}
\author{Frank James\thanks{email: frank.a.james@gmail.com}}

\begin{document}

\maketitle

\section{Algorithms and data structures}
Polynomials are implemented as flat arrays of integer coefficients. The coefficients may be fixed precision integers (fixnums)
or high precision integers (bignums).

Coefficients in the array can be accessed using the function {\tt offset} defined below. 
\begin{lstlisting}
  (defun base-offset (num-vars degree)
    (if (< degree 0)
        0
        (ncr (+ num-vars degree)
             degree)))

  (defun power-offset (powers)
    (let ((degree (reduce #'+ (cdr powers))))
      (cond
        ((or (null (cdr powers))
             (= (+ (car powers) degree) 0))
         0)
        (t
          (+ (number-terms (length powers) (1- degree))
          (power-offset (cdr powers)))))))

  (defun offset (powers)
    (+ (base-offset (length powers) (1- (reduce #'+ powers)))
       (power-offset powers)))
\end{lstlisting}

Note that the total number of coefficients $N$ is given by
\begin{equation}
  N = \frac{(n + m)!}{n!},
\end{equation}
with $n$ the degree of the polynomial and $m$ the number of variables.


\subsection{Systems of linear equations}
Systems of linear equations with polynomial coefficients can be solved using the fraction free Gaussian elimination (FFGE) algorthim.
This can be implemented for either integers or polynomials in general.

There are two approaches to implementing FFGE.
\begin{enumerate}
  \item
    Directly implementing FFGE for matrices of polynomial (or integer) entries
  \item
    Choosing a set of evaluation points for all the variables in the polynomial entries and then evaluating
    the matrix at each of these points. This results in a set of matrices of numbers only, which can then be
    solved using FFGE in the standard way. These can then be combined back into the polynomial solutions using
    Lagrange interpolation
\end{enumerate}

In both of these cases, bignum integers are required. This can be overcome by first choosing a set of prime numbers and taking
the input matrix modulo these. Then either of the above processes can be used to generate solution matrices for each of the primes.
Finally, these can be combined back into the complete solution using the Chinese Remainder theorem. Using this procedure, only
fixnum integers are required.


\section{Issues}

\subsection{Problems with polynomial FFGE}
The FFGE algorithm causes the degree of the polynomials to increase. The polynomial arithmetic
functions addition/subtraction scale as the total number of coefficients $N$ whereas multiplication
scales as $N^2$. The situation with division is less clear since it depends on the density of the dividend/divisor.

Let us see how $N$ scales as $n, m$,
\begin{equation}
  \frac{N_{n+1,m}}{N_{n,m}} = 1 + \frac{m}{n + 1}, \\
  \frac{N_{n,m+1}}{N_{n,m}} = \frac{n + m + 1}.
\end{equation}
Clearly increasing the degree of a polynomial only slowly increases the total number of coefficients. However,
increasing the number of variables very quickly increases the number of coefficients. For polynomials of reasonably
large degree ($~20$) this is clearly a very large increase.

\subsection{Probiems with evaluation/interpolation}
The FFGE algorithm scales as $r^3$ with $r$ the number of unknowns to solve for in the input linear system. This
is the rank of the input matrix.

Interpolation on the otherhand requires computing determinants of $N x N$ matrices. Determinants can be computed using
a fraction free LU decomposition algorithm, which is essentially a generalisation of FFGE, and also scales as $n^3$. However,
in this case, the rank of the input matrix is not $r$ but $N$. As shown above, $N$ can become very large rather quickly
and the work required to compute these determinants becomes far more than is required to reduce the input matrix.

\section{Conclusion}
Both approaches have their merits and unfortunately both suffer from drawbacks. In the case of the evaluation/interpolation
approach, the drawback seems rather serious. 


\end{document}
